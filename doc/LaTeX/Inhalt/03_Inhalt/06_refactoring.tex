\chapter{Refactoring}

\section{Code Smells identifizieren}
Prägnante Fälle von Shotgun Surgery und Large Classes sind im Code des Fahrtkostenrechners praktisch nicht vorhanden, was auch darin begründet liegt, dass das Projekt relativ klein ist.
Switch Statements wurden in der Codebasis nicht verwendet und mit Kommentaren war der bearbeitende Student auch sparsam.
Nichtsdestotrotz finden sich im Code-Substrat einige Code Smells, welche größtenteils aus Duplicated Code und Long Methods bestehen.

Insbesondere im GUI Plugin finden sich sehr viele \emph{Code-Dopplungen}.
Die wohl prominenteste dieser Dopplungen entsteht durch die \href{https://github.com/yschiebelhut/carpool-java/blob/6a206569e676ad1bf55d68ba3e64408bae83e73f/0-carpool-java-plugin-ui/src/main/java/gui/MainGUI.java#L42}{Erzeugung von Popups}.
Diese wurde so umgesetzt, dass das GUI-Fenster, aus dem das Popup hervorgeht, gesperrt wird, wenn das Popup gestartet wird.
Wird das Popup geschlossen, so wird auch das Ursprungsfenster wieder freigegeben.

\emph{Long Methods} finden sich praktisch in allen GUI-Klassen abgesehen vom Controller.
Alle GUI-Klassen folgen grob demselben schematischen Aufbau, bei dem sämtliche Elemente im Konstruktur der jeweiligen Ansicht aufgebaut werden.
Das macht zum einen den Konstruktor sehr unübersichtlich, andererseits sind die Ansichten auch sehr unflexible gegenüber nachträglichen Änderungen.
Das Anzeigen neuer Elemente oder das Aktualisieren der angezeigten Daten ist meist nur über einen erneuten Aufruf des Konstruktors möglich.

Auch in der Klasse \code{Telegram} im Integrations-Plugin findet sich ein Code-Smell, der sich trotz kompakter Erscheinung der Methode am ehesten als \emph{Long Method} einordnen lässt.
Hauptproblem ist hier, dass sehr viele verschiedene Aktionen ausgeführt werden, die thematisch nicht zusammenhängen, was die Lesbarkeit stark beeinträchtigt.

Eine weitere lange und unübersichtliche Methode findet sich in der Klasse \href{https://github.com/yschiebelhut/carpool-java/blob/379784c1ffe99a3f4fc15f393e12661479e6a4bf/3-carpool-java-domain/src/main/java/model/Fahrperiode.java#L85}{\code{Fahrperiode}}.
In dieser Periode werden die Kosten für jeden Mitfahrer innerhalb einer Fahrperiode berechnet.
In der Folge enthält diese Methode mehrere ineinander geschachtelte Schleifenaufrufe, die über alle Fahrten in der Periode und über alle jeweils beteiligten Mitfahrer iterieren.
Auch diese Methode ist kaum lesbar und bedarf dringend einer Verbesserung.

\section{Refactorings anwenden}
\begin{itemize}
    \item Extract Method
    \item Rename Method
    \item Replace Temp with Query
    \item Replace Conditional with Polymorphism
    \item Replace Error Code with Exception
    \item Replace Inheritance with Delegation
\end{itemize}


Telegram: Long Method > Method extraction, Replace Temp with Query

ErgebnisGUI: Long Method
FahrgemeinschaftGUI: Long Method

Duplicated Code (PopUp):
MainGUI
FahrgemeinschaftGUI
FahrperiodeGUI
MitgleiderFahrgemeinschaftGUI
